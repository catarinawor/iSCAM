%
%  Documentation on TMA development
%
%  Created by Steven James Dean Martell on 2011-12-13.
%  Copyright (c) 2011 UBC Fisheries Centre. All rights reserved.
%

% taken most packages from iSCAM manual
\documentclass[12pt,letterpaper]{article}

% Use utf-8 encoding for foreign characters
\usepackage[utf8]{inputenc}

% Setup for fullpage use
\usepackage{fullpage}
\usepackage{lscape}


% Multicols
\usepackage{multicol}
\setlength{\columnseprule}{0.5pt}
\setlength{\columnsep}{15pt}

% More symbols
\usepackage{amsmath}
\usepackage{amssymb}
\usepackage{latexsym}
\usepackage{bm}

%write pseudocode
\usepackage{algorithm}
\usepackage{algorithmic}

% Surround parts of graphics with box
\usepackage{boxedminipage}

% Longtables
\usepackage{longtable}

% Package for including code in the document
\usepackage{listings}
\usepackage{alltt}


% This is a custom iscamx package with the logo defs and fancyhdr
\usepackage{iscamx}


% If you want to generate a toc for each chapter (use with book)
% \usepackage{minitoc}

% Natbib
\usepackage[round]{natbib}


% Hyperref
% \usepackage{url}
\usepackage[colorlinks,bookmarks,citecolor=magenta,linkcolor=blue]{hyperref}
% \usepackage{hyperref}




\title{Concept and definition of mortality per recruit and fisheries footprint}

\author{Steven J. D. Martell and Catarina Wor\\
International Pacific Halibut Commission\\
2320 West Commodore Way,\\
Seattle, WA\\
98199-1287\\
USA\\
\texttt{stevem@iphc.int}
}

\date{\today}


\begin{document}

\section{Rationale}

Exploited resources are commonly partitioned among the many user sectors in terms of yield. That is, a portion of the biomass, in terms of weight, is distributed (or allocated) according to pre-defined rules and agreements. One problem with yield allocation is that it doesn't necessarily generate the right incentives for all users to increase the overall yield per recruit of the fisheries. This happens because the individual sectors are not penalized for targeting very small fish, which could potentially lead to growth overfishing. For example, one could imagine a fish stock being exploited by two sectors with equal yield allocation, but no restrictions regarding gear selectivity. Sector 1 targets large adults while the sector 2 targets juveniles. Even though both sectors are taking the same biomass, it is likely that sector 2 has a much larger impact on the long term spawning biomass as they would need to harvest a much larger number of individuals to reach their share of the total yield. 

 We propose the use of number of fish instead of weight as a measure of the fisheries footprint for each user of a fisheries resource.  We define fisheries footprint as the proportion of the total mortality per recruit inflicted by a given user group. In addition, mortality per recruit is defined as the total number of fish harvested by a given user group divided by the average number of recruits generated by the current biomass level.  


\section{Model description}

In order to calculate the fisheries footprint we use a age structured equilibrium population dynamics model. 


%Index tables

\begin{longtable}{ll}

\caption[index symbols]{A list of symbols used for indexing and naming model variables.} 
\label{tab:list_of_symbols} \tabularnewline

\hline 
\multicolumn{1}{l}{\textbf{Symbol}} & 
\multicolumn{1}{l}{\textbf{Description}} \\
\hline\\[-2ex]
\endfirsthead

\multicolumn{2}{l}{\underline{Indexes}}\\
$k$  & index for gear  \\      
$a$  & index for age \\      
$e$  & index for exploited equilibrium  \\      
$0$  & index for unexploited equilibrium \\ 
$q$  & index for yield per recruit \\
$m$  & index for mortality per recruit \\
\hline\\[-2ex]
\multicolumn{2}{l}{\underline{Symbols}}\\
$y_{k}$ 			& gear-specific yield (in weight)\\
$m_{k}$ 			& gear-specific mortality (in numbers)\\
$f_{k}$ 			& gear-specific fishing mortality rate\\
$v_{a,k}$ 			& age- and gear-specific selectivity\\
$M_{a}$ 			& age-specific natural mortality rate \\
$Z_{a}$ 			& age-specific total mortality rate \\
$s_{a}$ 			& age-specific survival \\
$l_{a}$ 			& age-specific survivorship \\
$l_{a}$ 			& age-specific unfished survivorship \\
$R_e$     			& equilibrium recruitment.\\
$R_o$  	 			& equilibrium unfished recruitment.\\
$\Omega$			& recruitment compensation ratio.\\
$\phi_{e} $     	& spawning biomass per recruit.\\
$\phi_{0} $     	& unfished spawning biomass per recruit.\\
$w_{a}$ 			& age-specific weight \\
$mat_a$ 			& age-specific maturity at age \\
$\phi_{q}$ 			& \\
$\phi_{m}$ 			& \\
$q_{a,k}$ 			& \\
$p_{a,k}$ 			& \\
$a_k$ 				& gear-specific proportion of total harvest (in weight or numbers)\\
$\lambda_{k}$ 		& gear-specific fishing mortality rate allocation\\
\hline\\[-2ex]
\end{longtable}



\begin{table}[h]
\centering
\caption{Equilibrium population dynamics model required for fisheries footprint calculation}\label{tab:equilibrium_model} 
\tableEq
	\large
    \begin{align}
           \hline
        \mbox{Catch Equations:} \nonumber \\
        	y_{k} &=  f_{k} \cdot R_{e} \phi_{q,k}  \label{T2.1}\\
        	m_{k} &=  f_{k} \cdot R_{e} \phi_{m,k}   \label{T2.2}\\[1ex]
        \hline
        \mbox{Survivorship:} \nonumber \\
        	Z_a &= M_a + \sum_k f_k v_{k,a} \label{T2.3}\\[1ex]
        	s_a &= \exp(-Z_a) \label{T2.4}\\[1ex]
        	l_a &= \begin{cases}
				1,  & a = 1\\[1ex]
				l_{a-1} s_{a-1}, & 1<a<A\\[1ex]
				\dfrac{l_{a}}{1 - s_{a}}, & a = A
			\end{cases} \label{T2.5}\\[1ex]        	
        \hline
        \mbox{Recruitment:} \nonumber \\
        	R_e &= R_o \frac{(\Omega-\phi_{0}/\phi_{e})}{(\Omega - 1)}\label{T2.6}\\[1ex]   
        \hline
        \mbox{Per Recruit Functions:} \nonumber \\
        	\phi_{e} &= \sum_a l_a \cdot w_a \cdot mat_a\label{T2.7}\\[1ex]
        	\phi_{0} &= \sum_a l_x \cdot w_a \cdot mat_a\label{T2.8}\\[1ex]
        	q_{a,k} &= w_a \cdot v_{a,k} \cdot (1-s_a)/ Z_a \label{T2.9}\\[1ex]
			{\phi_{q}}_{k} &= \sum_a l_a \cdot q_{a,k} \label{T2.10}\\[1ex]
			p_{a,k} &= v_{a,k} \cdot (1-s_a)/ Z_a \label{T2.11}\\[1ex]
			{\phi_{m}}_{k} &= \sum_a l_a \cdot p_{a,k} \label{T2.12}\\       
        \hline \hline \nonumber
    \end{align}
\end{table}





\subsection{Allocation and target $F_{SPR}$ calculation}

The allocation parameter ($\lambda_k$) can be calculated by three alternative methods, depending on variable chosen to define the partition agreement. The distribution of the resource can be based on proportions of target yield or proportion of target mortality. A third option is generated when a fixed yield is allocated to a given sector.


\begin{table}[h!]
\centering
\caption{Allocation Equations}\label{tab:allocation} 
\tableEq
	\large
    \begin{align}
    \hline
    	\mbox{f Allocation:} \nonumber \\
		        	f_{k} &=  f^{\star} \cdot \lambda_{k}  \label{T3.13}\\ 
		\hline        	
		\mbox{Yield Allocation:} \nonumber \\
					\lambda_k &= \dfrac{a_k}{{\phi_q}_k / \sum_k {\phi_q}_k } \label{T3.1}\\[1ex]
		\hline 
		\mbox{Mortality Allocation:} \nonumber \\
					\lambda_k &= \dfrac{a_k}{{\phi_m}_k / \sum_k {\phi_m}_k } \label{T3.2}\\[1ex]
		\hline 
		\mbox{Spawning Potential Ratio:} \nonumber \\
					SPR &= \dfrac{\phi_{e}}{\phi_{0}} \label{T3.3}\\[1ex]
\hline \hline \nonumber   
   \end{align}
\end{table}

The calculation of $\lambda_k$ has to be done iteratively for a given $F_{SPR}$ target as the per recruit quantities depend on $f_k$ and consequently $\lambda_k$. We find that a maximum of four iterations are sufficient for the model to reach convergence.  

In order to determine the target $F^{\star}$ one need a previously determined target Spawning Potential Ratio (${SPR}_{target}$). Once the ${SPR}_{target}$ is chosen, $F^{\star}$ can be calculated as the fishing mortality rate that would lead to $SPR = {SPR}_{target}$. The pseudo code to calculate $F^{\star}$ is: 


\begin{algorithm} 
\caption{Calculate $F^{\star}$}
\begin{algorithmic} 
\large
\STATE 1 - $F^{\star} \leftarrow F^{\star}_{guess}$
\STATE 2 - calculate $SPR$ with the equilibrium model
\STATE 3 - obj function $\leftarrow (SPR-SPR_{target})^2$
\STATE 4 - minimize the objective function by changing $F^{\star}$
\end{algorithmic}
\end{algorithm}

The algorithm for calculating $F^{\star}$ has to be modified when one a fixed yield allocation policy is in place. In that scenario one or more sector gets a set yield value instead of a proportion of the total yield or mortality.


\begin{algorithm} 
\large
\caption{Calculate $F^{\star}$}
\begin{algorithmic}
\STATE 1 - $bgear \leftarrow $ gear with fixed yield allocation 
\STATE 2 - $a_k[bgear] \leftarrow {a_k[bgear]}_{guess}$ 
\STATE 3 - $p_k \leftarrow a_k[!bgear]/\sum_k a_k[!bgear]$ 
\STATE 4 - $a_k[!bgear] \leftarrow (1-\sum_k a_k[bgear])* pk$
\STATE 5 - $F^{\star} \leftarrow F^{\star}_{guess}$
\STATE 6 - calculate $SPR$ and $yield[bgear]$ with the equilibrium model
\STATE 7 - obj function $\leftarrow (SPR-SPR_{target})^2 + (ye[bgear]-{ye[bgear]}_{target})^2$
\STATE 8 - minimize the objective function by changing $F^{\star} and a_k[bgear] $
\end{algorithmic}
\end{algorithm}   




\end{document}


